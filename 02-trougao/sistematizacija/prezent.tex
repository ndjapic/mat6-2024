\documentclass[11pt]{beamer}
%\usetheme{CambridgeUS}
\usetheme{Berkeley}
\usepackage[OT2]{fontenc}
\usepackage[utf8x]{inputenc}
\usepackage[serbian]{babel}
\usepackage{amsmath,amssymb}
\usepackage{tikz}
%\usepackage{graphicx}

\title{Троугао}
\author{систематизација}
\date{\today}

\begin{document}

\begin{frame}
\titlepage
\end{frame}

\begin{frame}
\frametitle{Увод}
У овој презентацији ћемо се подсетити свега што смо учили о троуглу. Погледаћемо збир углова, врсте углова, однос страница и углова, и још много тога.
\end{frame}

\section{Понављање основних појмова}
\begin{frame}{Елементи троугла}

\onslide<1->
\begin{tikzpicture}
\draw (0,0) -- (6,0) -- (3,4) -- cycle;
\draw (0,0) node[below left] {$\mathnormal{A}$};
\draw (6,0) node[below right] {$\mathnormal{B}$};
\draw (3,4) node[above] {$\mathnormal{C}$};
\draw (1,0) arc (0:53:1) node[midway,right] {$\alpha$};
\draw (5,0) arc (180:127:1) node[midway,left] {$\beta$};
\draw (3.75,3) arc (318:222:1) node[midway,below] {$\gamma$};
\end{tikzpicture}

\onslide<2->
\begin{itemize}
    \item Темена: $\mathnormal{A}$, $\mathnormal{B}$, $\mathnormal{C}$
    \item Странице: $\mathnormal{AB}$, $\mathnormal{BC}$,
        $\mathnormal{AC}$
    \item Углови: $\alpha$, $\beta$, $\gamma$
\end{itemize}
\end{frame}

\section{Збир углова у троуглу}
\begin{frame}{Збир углова у троуглу}

\onslide<1->
\begin{itemize}
    \item Формула: $\alpha + \beta + \gamma = 180^\circ$
\end{itemize}

\onslide<2->
\textbf{Задатак 1:}
Ако су два угла у троуглу $45^\circ$ и $60^\circ$, колики је трећи угао?

\end{frame}

\begin{frame}\frametitle{Збир углова у троуглу}

\onslide<1->
\begin{block}{Теорема}
Збир углова у сваком троуглу је 180 степени.
\end{block}

\onslide<2->
\textbf{Задатак 2:}
Ако су два угла у троуглу 60° и 80°, колико је трећи угао?

\onslide<3->{
\textbf{Решење:} 
Трећи угао је $180^\circ - 60^\circ - 80^\circ = 40^\circ$.
}

\onslide<4->
\textbf{Задатак 3:}
У једном троуглу, један угао је за 30° већи од другог, а трећи је двоструко већи од најмањег. Колики су сви углови тог троугла?

\onslide<5->
\textbf{Задатак 4:}
Може ли троугао имати углове од 50°, 60° и 80°? Објасни свој одговор.

\end{frame}

% ... остатак презентације са осталим темама и задацима ...

\section{Углови на трансверзали и углови троугла}
\begin{frame}{Углови на трансверзали и углови троугла}

\onslide<1->
%\begin{tikzpicture}
%% ... (нацртати паралелне праве и трансверзалу)
%\end{tikzpicture}
\begin{figure}[!ht]
\centering
\resizebox{0.2\textwidth}{!}{%
\begin{tikzpicture}
\tikzstyle{every node}=[font=\Large]
\draw (0,1) -- (5,1);
\draw (0,2) -- (5,2);
\draw (1,0) -- (4,3.5);
\draw (0,1) node[below right] {$\mathnormal{a}$};
\draw (0,2) node[above right] {$\mathnormal{b}$};
\draw (4,3.5) node[below right] {$\mathnormal{t}$};
\end{tikzpicture}
}%
\caption{Трансверзала пресеца две паралеле}
\label{transverzala}
\end{figure}

\onslide<2->
\textbf{Задатак 5:}
Ако су два угла на истој страни трансверзале и између паралелних правих $70^\circ$ и $110^\circ$, колики су углови у троуглу који се образује?

\onslide<3->
\textbf{Задатак 6:}
Ако су две паралелне праве пресечене трансверсалом тако да је један од насталих унутрашњих наизменичних углова 75°, колики су сви остали унутрашњи углови?

\onslide<4->
\textbf{Задатак 7:}
Нацртај две паралелне праве и једну трансверсалу. Означи све одговарајуће углове и њихове односе.

\end{frame}

\section{Однос страница и углова троугла}
\begin{frame}{Однос страница и углова троугла}

\onslide<1->
\begin{itemize}
    \item Насупрот већег угла лежи већа страница.
    \item Насупрот једнаких углова леже једнаке странице.
\end{itemize}

\onslide<2->
\textbf{Задатак 8:}
У троуглу $\mathnormal{ABC}$ је $\alpha > \beta$.
Која је од страница
$\mathnormal{BC}$ или $\mathnormal{AC}$ већа? Зашто?

\onslide<3->
\textbf{Задатак 9:}
У троуглу $\mathnormal{ABC}$, страница $\mathnormal{AB}$ је најдужа.
Који је највећи угао у том троуглу? Објасни.

\onslide<4->
\textbf{Задатак 10:}
Може ли троугао имати две странице дужине
$5 \,\mathrm{cm}$ и $7 \,\mathrm{cm}$,
а трећа страница да буде $13 \,\mathrm{cm}$? Зашто?

\end{frame}

% ... (Nastaviti sa ostalim temama i zadacima)

\section{Неједнакост троугла}
\begin{frame}{Неједнакост троугла}

\onslide<1->
Страница троугла је
\begin{itemize}
    \item краћа од збира друге две странице,
    \item а дужа од њихове разлике.
\end{itemize}

\onslide<2->
\textbf{Задатак 11:}
Да ли је могуће конструисати троугао са страницама дужине
$3 \,\mathrm{cm}$, $5 \,\mathrm{cm}$ и $8 \,\mathrm{cm}$? Објасни.

\onslide<3->
\textbf{Задатак 12:}
Ако су две странице троугла дужине
$4 \,\mathrm{cm}$ и $6 \,\mathrm{cm}$,
између којих вредности мора бити дужина треће странице?

\end{frame}

\section{Странице и углови троугла}
\begin{frame}{Странице и углови троугла}

\onslide<1->
\textbf{Задатак 13:}
У једнакокраком троуглу, основица је дужине
$8 \,\mathrm{cm}$, а краци су дужине $5 \,\mathrm{cm}$.
Који је највећи угао у том троуглу?

\onslide<2->
\textbf{Задатак 14:}
У једнакостраничном троуглу, сви углови су једнаки. Колика је мера сваког угла?

\end{frame}

\section{Врсте троуглова и основна својства}
\begin{frame}{Врсте троуглова и основна својства}

\onslide<1->
\textbf{Задатак 15:}
Наведи све врсте троуглова које знаш и опиши њихове особине.

\onslide<2->
\textbf{Задатак 16:}
Нацртај један оштроугли, један правоугли и један тупоугли троугао.
Означи њихове углове.

\end{frame}

\end{document}
