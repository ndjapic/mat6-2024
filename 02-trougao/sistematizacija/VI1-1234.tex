\documentclass[10pt,a5paper,addpoints]{exam}

\usepackage{myPaper}
\usepackage{multicol}
\usepackage{amssymb,amsmath}
\usepackage{epsf}

\usepackage[OT2]{fontenc}
\usepackage[utf8x]{inputenc}
\usepackage[serbian]{babel}

\def\grupa#1#2#3#4{#1}
\title{$\mathrm{VI}$ разред, група \grupa 1234}
\author{Први писмени задатак
 \thanks{
  18 одлично,
  14 врло добро,
  10 добро,
   5 довољно.
 }
}
\date{Станишић, 25.\ октобар 2024.}
%\date{}

\printanswers
%\renewcommand{\solutiontitle}{\noindent\textrm{Решење:}\enspace}
\renewcommand{\solutiontitle}{}
\pointsinmargin
%\pointsinrightmargin
\pointname{}
%\marginpointname{\%}
%\pagename{Страница}
\hqword{Задатак:}
\hpgword{Страница:}
\hpword{Поени:}
\hsword{Остварено:}
\htword{Збир}
\cellwidth{1em}
%\gradetablestretch{1.1}
\cfoot[]{Страница \thepage\ од \numpages}

\def\abs|#1|{\left| #1 \right|}

\hyphenation{
}

\begin{document}

\maketitle
\thispagestyle{headandfoot}

%\vspace*{\stretch 1}
\noindent \gradetable[h]

\begin{questions}

\question %01.
 Из скупа
 $M = \{
  \grupa
   {4,-6,-18,3,0,-5,2}
   {-4,6,18,-3,0,5,-2}
   {4,6,-18,-3,0,-5,2}
   {4,6,18,-3,0,-5,-2}
 \}$ издвој подскупове:
 \begin{parts}
%
 \part[1] $A$ \grupa{пози}{нега}{нега}{пози}тивних целих бројева.
 \begin{solution}
  $A = \{
   \grupa
    {4,3,2}
    {-4,-3,-2}
    {-18,-3,-5}
    {4,6,18}
  \}$.
 \end{solution}
%
 \part[1] $B$ \grupa{нега}{пози}{пози}{нега}тивних целих бројева.
 \begin{solution}
  $B = \{
   \grupa
    {-6,-18,-5}
    {6,18,5}
    {4,6,2}
    {-3,-5,-2}
  \}$.
 \end{solution}
%
 \end{parts}

\question %03.
 Израчунај:
% \begin{multicols}2
 \begin{parts}
 \part[1] $\grupa{(−7)+(+5)}{8+(−5)}{(−6)+8}{(−8)+(-2)}$
 \begin{solution}
  $
   \grupa
    {(−7)+(+5) = -(7-5) = -2}
    {8+(−5) = +(8-5) = 3}
    {(−6)+8 = +(8-6) = 2}
    {(−8)+(-2) = -(8+2) = -10}
  $.
 \end{solution}
 \part[1] $\grupa{(-2)-(-7)}{(+4)-(-6)}{5-(-3)}{(-2)-(+4)}$
 \begin{solution}
  $
   \grupa
    {(-2)-(-7) = -2+7 = +(7-2) = 5}
    {(+4)-(-6) = 4+6 = 10}
    {5-(-3) = 5+3 = 8}
    {(-2)-(+4) = -2-4 = -(2+4) = -6}
  $.
 \end{solution}
 \part[1] $\grupa{(-3)⋅(-5)}{(-2)⋅(+4)}{(+2)⋅(-3)}{(-5)⋅(-2)}$
 \begin{solution}
  $
   \grupa
    {(-3)⋅(-5) = -(-( 3⋅5 )) = 15}
    {(-2)⋅(+4) = -(+( 2⋅4 )) = -8}
    {(+2)⋅(-3) = +(-( 2⋅3 )) = -6}
    {(-5)⋅(-2) = -(-( 5⋅2 )) = 10}
  $.
 \end{solution}
 \part[1] $\grupa{(-16):(+4)}{12:(-4)}{(-15):(-3)}{(-12):(-3)}$
 \begin{solution}
  $
   \grupa
    {(-16):(+4) = -(+( 16:4 )) = -4}
    {12:(-4) = -( 12:4 ) = -3}
    {(-15):(-3) = -(-( 15:3 )) = 5}
    {(-12):(-3) = -(-( 12:3 )) = 4}
  $.
 \end{solution}
 \end{parts}
% \end{multicols}

\question %02.
 Испитај да ли постоји троугао:
 \begin{parts}
 \part[1] чије су странице
  $\grupa 5445 \,\mathrm{cm}$,
  $\grupa 3553 \,\mathrm{cm}$ и
  $\grupa 9889 \,\mathrm{cm}$;
 \begin{solution}
  \grupa
   {$3<5<9$,\quad $3+5=8<9$,\quad не постоји}
   {$4<5<8$,\quad $4+5=9>8$,\quad постоји}
   {$4<5<8$,\quad $4+5=9>8$,\quad постоји}
   {$3<5<9$,\quad $3+5=8<9$,\quad не постоји}.
 \end{solution}
 \part[1] чији су унутрашњи углови
  $\grupa{45}{65}{45}{65}^\circ$, $75^\circ$ и
  $\grupa{60}{50}{60}{50}^\circ$.
 \begin{solution}
  $
   \grupa{45}{65}{45}{65}^\circ +
   75^\circ +
   \grupa{60}{50}{60}{50}^\circ =
   \grupa{120}{140}{120}{140}^\circ +
   \grupa{60}{50}{60}{50}^\circ =
   \grupa{180}{190}{180}{190}^\circ
  $,\quad 
  \grupa{постоји}{не постоји}{постоји}{не постоји}.
 \end{solution}
 \end{parts}

\question[2] %04.
    Препиши у вежбанку и допуни наредну реченицу.
    \grupa{
        У једнакокраком троуглу
        једнаке странице називамо %\hfill\break
        \underline{\Large\phantom{ краци }},
        а трећа страница је %\hfill\break
        \underline{\Large\phantom{ основица }}
        тог троугла.
        \begin{solution}
            У једнакокраком троуглу
            једнаке странице називамо %\hfill\break
            \underline{\Large{ краци }},
            а трећа страница је %\hfill\break
            \underline{\Large{ основица }}
            тог троугла.
        \end{solution}
    }{
        У правоуглом троуглу
        највећу страницу називамо %\hfill\break
        \underline{\Large\phantom{ хипотенуза }},
        а две мање странице су %\hfill\break
        \underline{\Large\phantom{ катете }}
        тог троугла.
        \begin{solution}
            У правоуглом троуглу
            највећу страницу називамо %\hfill\break
            \underline{\Large{ хипотенуза }},
            а две мање странице су %\hfill\break
            \underline{\Large{ катете }}
            тог троугла.
        \end{solution}
    }{
        Троугао чији је највећи угао туп називамо %\hfill\break
        \underline{\Large\phantom{ тупоугли }}
        троугао, а троугао који има три оштра угла је %\hfill\break
        \underline{\Large\phantom{ оштроугли }}
        троугао.
        \begin{solution}
            Троугао чији је највећи угао туп називамо %\hfill\break
            \underline{\Large{ тупоугли }}
            троугао, а троугао који има три оштра угла је %\hfill\break
            \underline{\Large{ оштроугли }}
        \end{solution}
    }{
        Троугао чије су све странице једнаке дужине
        називамо\hfill\break
        \underline{\Large\phantom{ једнакостранични }}
        троугао, а троугао чије су две странице једнаке дужине
        је %\hfill\break
        \underline{\Large\phantom{ једнакокраки }}
        троугао.
        \begin{solution}
            Троугао чије су све странице једнаке дужине
            називамо\hfill\break
            \underline{\Large{ једнакостранични }}
            троугао, а троугао чије су две странице једнаке дужине
            је %\hfill\break
            \underline{\Large{ једнакокраки }}
            троугао.
        \end{solution}
    }

\question %05.
 Дати су цели бројеви:
 \grupa
  {$11$, $12$, $-4$, $-21$, $-8$, $15$, $-10$}
  {$-12$, $-4$, $-8$, $15$, $11$, $21$, $-10$}
  {$11$, $-8$, $12$, $-4$, $-21$, $15$, $-10$}
  {$-8$, $15$, $11$, $-12$, $-4$, $21$, $-10$}.
 \begin{parts}
 \part[1] Израчунај \grupa{збир}{разлику}{збир}{разлику}
  нај\grupa{већ}{већ}{мањ}{мањ}ег и
  нај\grupa{мањ}{мањ}{већ}{већ}ег од датих бројева.
  \begin{solution}
   $
    \grupa
     {15+(-21) = -(21-15) = -6}
     {21-(-12) = 21+12 = 33}
     {(-21)+15 = -(21-15) = -6}
     {(-12)-21 = -(12+21) = -33}
   $.
  \end{solution}
 \part[1] Поређај дате бројеве од
  нај\grupa{мањ}{већ}{већ}{мањ}ег ка
  нај\grupa{већ}{мањ}{мањ}{већ}ем.
  \begin{solution}
   $
    \grupa
     {-21 < -10 < -8 < -4 < 11 < 12 < 15}
     {21 > 15 > 11 > -4 > -8 > -10 > -12}
     {15 > 12 > 11 > -4 > -8 > -10 > -21}
     {-12 < -10 < -8 < -4 < 11 < 15 < 21}
   $
  \end{solution}
 \end{parts}

\question[3] %03.
 Израчунај вредност израза:
 $
  \grupa
  {-14 + 6 ⋅ (-3) - (-48) : (-6)}
  {-15 + (-52) : (-4) - 8 ⋅ (-3)}
  {-17 - 9 ⋅ (-3) + (-68) : (-4)}
  {-16 - (-45) : (-9) + 6 ⋅ (-3)}
 $
 \begin{solution}
  \begin{align*}
   \grupa
    {-14 + 6 ⋅ (-3) - (-48) : (-6)}
    {-15 + (-52) : (-4) - 8 ⋅ (-3)}
    {-17 - 9 ⋅ (-3) + (-68) : (-4)}
    {-16 - (-45) : (-9) + 6 ⋅ (-3)}
   &= \grupa{-14 -18 -8}{-15 +13 +24}{-17 +27 +17}{-16 -5 -18}
   \\ &= \grupa{-32 -8}{-2 +24}{-44 +17}{-21 -18}
   \\ &= \grupa{-40}{22}{-27}{-39}
  \end{align*}
 \end{solution}

\question[2] %09.
 Угао \grupa{на основици}{при врху}{на основици}{при врху}
 једнакокраког троугла
 $\triangle ABC$ ($AC=BC$) је $\grupa 7557 0^\circ$.
 Упореди дужине крака и основице тог троугла.
 \begin{solution} \centering
  \grupa{
   $\alpha = \beta = 70^\circ$,\qquad
   $\gamma = 180^\circ - 2 ⋅ 70^\circ
   = 180^\circ - 140^\circ = 40^\circ$,\qquad
   $\alpha > \gamma$,\qquad
   $BC > AB$,\qquad
   крак је дужи од основице.
  }{
   $\gamma = 50^\circ$,\qquad
   $\alpha + \beta = 180^\circ - 50^\circ = 130^\circ$,\qquad
   $\alpha = 130^\circ : 2 = 65^\circ$,\qquad
   $\alpha > \gamma$,\qquad
   $BC > AB$,\qquad
   крак је дужи од основице.
  }{
   $\alpha = \beta = 50^\circ$,\qquad
   $\gamma = 180^\circ - 2 ⋅ 50^\circ
   = 180^\circ - 100^\circ = 80^\circ$,\qquad
   $\alpha < \gamma$,\qquad
   $BC < AB$,\qquad
   крак је краћи од основице.
  }{
   $\gamma = 70^\circ$,\qquad
   $\alpha + \beta = 180^\circ - 70^\circ = 110^\circ$,\qquad
   $\alpha = 110^\circ : 2 = 55^\circ$,\qquad
   $\alpha < \gamma$,\qquad
   $BC < AB$,\qquad
   крак је краћи од основице.
  }
 \end{solution}

\question %08.
 Дате су странице троугла
 $a = \grupa{12}{10,\!5}{10}{11,\!5} \,\mathrm{cm}$ и
 $\grupa bccb = \grupa{4,\!5}4{6,\!5}5 \,\mathrm{cm}$.
 У којим гра\-ни\-ца\-ма је:
 \begin{parts}
 \part[2] трећа страница;
 \begin{solution}
  \begin{gather*}
   a-b < c < a+b \\
   \grupa{12}{10,\!5}{10}{11,\!5} - \grupa{4,\!5}4{6,\!5}5 < c <
   \grupa{12}{10,\!5}{10}{11,\!5} + \grupa{4,\!5}4{6,\!5}5 \\
   \grupa{7,\!5}{6,\!5}{3,\!5}{6,\!5} < c <
   \grupa{16,\!5}{14,\!5}{16,\!5}{16,\!5} \\
   \text{Трећа страница је између }
   \grupa{7,\!5}{6,\!5}{3,\!5}{6,\!5} \,\mathrm{cm}
   \text{ и }
   \grupa{16,\!5}{14,\!5}{16,\!5}{16,\!5} \,\mathrm{cm}.
  \end{gather*}
 \end{solution}
 \part[2] обим троугла?
 \begin{solution}
  \begin{gather*}
   {\cal O} = a+b+c \\
   a+b + 7,\!5 < {\cal O} <
   a+b + \grupa{16,\!5}{14,\!5}{16,\!5}{16,\!5} \\
   \grupa{16,\!5}{14,\!5}{16,\!5}{16,\!5} +
   \grupa{7,\!5}{6,\!5}{3,\!5}{6,\!5} < {\cal O} <
   \grupa{16,\!5}{14,\!5}{16,\!5}{16,\!5} +
   \grupa{16,\!5}{14,\!5}{16,\!5}{16,\!5} \\
   \grupa{24}{21}{20}{23} < {\cal O} <
   \grupa{33}{29}{33}{33} \\
   \text{Обим је између } \grupa{24}{21}{20}{23} \,\mathrm{cm}
   \text{ и } \grupa{33}{29}{33}{33} \,\mathrm{cm}.
  \end{gather*}
 \end{solution}
 \end{parts}

\end{questions}

\end{document}
%\{} ^2
% $2,\!5 \,\mathrm{cm}$
% \begin{multicols}2
