\documentclass[11pt]{beamer}
\usetheme{Copenhagen}
%\usetheme{Madrid} % Изаберите жељену тему
%\usetheme{Darmstadt}
\usefonttheme[onlymath]{serif}
\usepackage[OT2]{fontenc}
\usepackage[utf8x]{inputenc}
\usepackage[serbian]{babel}
\usepackage{amsmath,amsfonts,amssymb}
\usepackage{graphicx,tikz}

\author{Рационални бројеви}
\title{Поредак и основне рачунске операције у скупу рационалних бројева}

%\setbeamercovered{transparent}
%\setbeamertemplate{navigation symbols}{}
%\logo{}
\institute{ОШ „Иван Горан Ковачић“, Станишић}
\date{12. децембар 2024.}
\subject{Математика 6}

\begin{document}

\begin{frame}
\titlepage
\end{frame}

\begin{frame}
\tableofcontents
\end{frame}

\section{Писмени задатак}

\subsection{Пример писменог задатка}
\begin{frame}{Писмени задатак из математике}
    \begin{enumerate}
        \item Пореди следеће рационалне бројеве:
            \begin{enumerate}
                \item $\frac{2}{3}$ и $\frac{5}{6}$
                \item $-\frac{1}{4}$ и $-\frac{3}{8}$
                \item $1\frac{1}{2}$ и $\frac{5}{3}$
            \end{enumerate}
        \item Израчунај:
            \begin{enumerate}
                \item $\frac{1}{2} + \frac{3}{4} - \frac{1}{8}$
                \item $\frac{2}{5} \cdot \left(-\frac{3}{2}\right) : \frac{1}{10}$
            \end{enumerate}
        \item Реши једначину:
            \begin{enumerate}
                \item $x + \frac{3}{4} = -\frac{1}{2}$
                \item $\frac{2}{3} - x = \frac{5}{6}$
            \end{enumerate}
        \item Представи графички на бројевној правој:
            \begin{enumerate}
                \item $-\frac{1}{2}$, $\frac{3}{4}$, $-1\frac{1}{4}$
            \end{enumerate}
        \item Задатак са применом: 
            Један део баште је засејан поврћем, а остали цвећем. Поврће заузима $\frac{2}{5}$ дела баште. Који део баште је засејан цвећем?
    \end{enumerate}
\end{frame}

\subsection{Додатни задатак}
\begin{frame}{Додатни задатак}
    Реши следеће једначине:
    \begin{enumerate}
        \item $x - \frac{1}{3} = \frac{2}{5}$
        \item $\frac{3}{4} + x = -\frac{1}{6}$
    \end{enumerate}
\end{frame}

\section{Задаци}

\subsection{Задатак 1: Рационални бројеви}
\begin{frame}{Задатак 1}
    Дати су рационални бројеви:
    $a = -\frac{3}{4}$,
    $b = 1\frac{1}{2}$,
    $c = -0,\!25$.
    \begin{enumerate}
        \item Представи бројеве $a$ и $c$ у облику децималног броја.
        \item Упореди бројеве $a$, $b$ и $c$ и упиши одговарајући знак $(<, >, =)$.
        \item Израчунај: $a + b - c$.
    \end{enumerate}
\end{frame}

\subsection{Задатак 2: Једначине}
\begin{frame}{Задатак 2}
    Реши једначине:
    \begin{enumerate}
        \item $x + \frac{2}{3} = -\frac{1}{6}$
        \item $1,\!5 - x = 0,\!2$
    \end{enumerate}
\end{frame}

\begin{frame}{Задаци}
\begin{enumerate}
    \item Реши следеће једначине:
        \begin{enumerate}
            \item $x + 3 = 7$
            \item $5 - x = 2$
            \item $-2 + x = -5$
        \end{enumerate}
    \item Реши једначину:
        \[ \frac{3}{4}x - \frac{1}{2} = \frac{5}{8} \]
    \item Нађи број који када се повећа за $2,\!5$ даје $7$.
    \item Ако од броја одузмемо $\frac{1}{3}$,
        добијамо $\frac{2}{5}$. Који је тај број?
\end{enumerate}
\end{frame}

\begin{frame}
\frametitle{Задатак 1}
\begin{enumerate}
    \item Пореди следеће рационалне бројеве: 
        \[ \frac{3}{4}, -\frac{5}{6}, \frac{7}{8} \]
    \item Представи рационални број $-\frac{11}{3}$ у облику мешовитог броја.
    \item Израчунај:
        \[ \frac{2}{5} + \left(-\frac{3}{4}\right) \cdot \frac{8}{9} \]
\end{enumerate}
\end{frame}

\begin{frame}
\frametitle{Задатак 2}
Реши једначину:
\[ x + \frac{2}{3} = -\frac{1}{4} \]
\end{frame}

\begin{frame}
\frametitle{Задатак 3}
Један аутомобил пређе $\frac{3}{5}$ пута за $2$ сата.
Колико ће времена требати да пређе цео пут?
\end{frame}

\begin{frame}
\frametitle{Задатак 4}
Нацртај бројевну праву и означи на њој следеће бројеве:
$-\frac{1}{2}$, $1,\!5$, $-\frac{4}{3}$.
\end{frame}

\end{document}
