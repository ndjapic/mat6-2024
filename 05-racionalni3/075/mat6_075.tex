\documentclass[11pt]{beamer}
%\usetheme{Copenhagen}
%\usetheme{Madrid} % Изаберите жељену тему
\usetheme{Darmstadt}
\usefonttheme[onlymath]{serif}
\usepackage[OT2]{fontenc}
\usepackage[utf8x]{inputenc}
\usepackage[serbian]{babel}
\usepackage{amsmath,amsfonts,amssymb}
\usepackage{graphicx,tikz}

\title{Једначине и неједначине у скупу рационалних бројева}
\author{Припремање за контролни задатак}

%\setbeamercovered{transparent}
%\setbeamertemplate{navigation symbols}{}
%\logo{}
\institute{ОШ „Иван Горан Ковачић“, Станишић}
\date{11. фебруар 2025.}
\subject{Математика 6}

\begin{document}

\begin{frame}
    \titlepage
\end{frame}

\begin{frame}
    \frametitle{Упут{}ство}
    \begin{enumerate}
        \item Време за рад је 30 минута.
        \item Није дозвољено коришћење калкулатора.
        \item Сваки задатак носи 10 поена.
        \item Решења неједначина прикажи на бројевној правој.
    \end{enumerate}
\end{frame}

\begin{frame}
    \frametitle{Задаци}

    \begin{enumerate}
        \item Реши једначину: $3x + 5 = 14$
        \item Реши једначину: $\frac{2}{3}x - 1 = 5$
        \item Реши једначину: $0.5x + 2.5 = 7$
        \item Реши неједначину: $2x - 3 < 7$
        \item Реши неједначину: $\frac{1}{4}x + 2 \geqslant 5$
        \item Реши неједначину: $1.2x - 3.6 \leqslant 0$
    \end{enumerate}

\end{frame}

\begin{frame}
    \frametitle{Решења}

    \begin{enumerate}
        \item $3x + 5 = 14 \Rightarrow 3x = 9 \Rightarrow x = 3$
        \item $\frac{2}{3}x - 1 = 5 \Rightarrow \frac{2}{3}x = 6 \Rightarrow x = 9$
        \item $0.5x + 2.5 = 7 \Rightarrow 0.5x = 4.5 \Rightarrow x = 9$
        \item $2x - 3 < 7 \Rightarrow 2x < 10 \Rightarrow x < 5$
        \item $\frac{1}{4}x + 2 \geqslant 5 \Rightarrow \frac{1}{4}x \geqslant 3 \Rightarrow x \geqslant 12$
        \item $1.2x - 3.6 \leqslant 0 \Rightarrow 1.2x \leqslant 3.6 \Rightarrow x \leqslant 3$
    \end{enumerate}

\end{frame}

\begin{frame}
    \frametitle{Упут{}ство}
    \begin{itemize}
        \item Време за рад: 30 минута.
        \item Сваки задатак носи одређени број поена (наведено поред задатка).
        \item Уредно напишите поступак решавања.
        \item Решења неједначина прикажи на бројевној правој.
        \item Користите хемијску оловку, изузев цртања.
    \end{itemize}
\end{frame}

\begin{frame}
    \frametitle{Задаци}

    \begin{enumerate}
        \item (2 поена) Реши једначину: $x + 3.5 = 7.2$
        \item (3 поена) Реши једначину: $\frac{2}{3}x - 1 = 5$
        \item (4 поена) Реши неједначину: $2x - 5 < 3$
        \item (5 поена) Реши неједначину: $\frac{1}{2}x + 2 \geqslant 4.5$
        \item (6 поена) Реши једначину: $3(x - 2) + 4 = 10$
        \item (7 поена) Реши неједначину: $2(x + 1) - 3 < 5$
    \end{enumerate}

\end{frame}

\begin{frame}
    \frametitle{Завршетак}
    Хвала на раду!
\end{frame}

\end{document}
