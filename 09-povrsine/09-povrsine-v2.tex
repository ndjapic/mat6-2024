\documentclass{beamer}
\usetheme{Copenhagen}
\usecolortheme{beaver}
\usefonttheme[onlymath]{serif}

\usepackage[utf8]{inputenc}
\usepackage[T2A]{fontenc} % Подршка за ћирилицу
%\usepackage{lmodern} % Коришћење modern фонтова
\usepackage{amsmath}
\usepackage{amsfonts}
\usepackage{amssymb}
\usepackage{graphicx}
\usepackage{tikz} % Пакет за цртање
\usetikzlibrary{shapes, arrows, positioning, graphs, graphdrawing} % Библиотеке за TikZ
\usegdlibrary{trees} % За распоређивање чворова у графу

\title{Површине троуглова и четвороуглова}
\author{Ваше Име} % Замените са својим именом
\date{\today} % Данашњи датум
\institute{ОШ „Иван Горан Ковачић“, Станишић}
\date{28. мај 2025.}
\subject{Математика 6}

% Дефиниција стилова за чворове у графу
\tikzstyle{formula} = [rectangle, rounded corners, draw=blue!80, fill=blue!20, very thick, minimum size=10mm, text width=3cm, align=center]
\tikzstyle{variable} = [circle, draw=green!80, fill=green!20, very thick, minimum size=8mm, align=center]
\tikzstyle{edge_style} = [thick, ->, >=stealth]

\begin{document}

\frame{\titlepage}

\begin{frame}
\frametitle{Увод 📐📏}
Добродошли на час математике! Данас ћемо истраживати како да израчунамо површине различитих геометријских фигура које већ познајете: троуглова и четвороуглова.
\begin{itemize}
    \item Подсетићемо се основних врста троуглова и четвороуглова.
    \item Научићемо формуле за израчунавање њихових површина.
    \item Видећемо како су те формуле повезане са страницама и висинама фигура.
\end{itemize}
Спремите се за путовање у свет облика и бројева!
\end{frame}

% --- Секција: Површина троугла ---
\section{Површина троугла}

\begin{frame}
\frametitle{Површина троугла 1/2}
Површина троугла представља величину дела равни коју тај троугао заузима.
\begin{itemize}
    \item Основна формула за површину троугла укључује дужину једне странице (основице) и дужину одговарајуће висине.
\end{itemize}
\begin{figure}
    \centering
    \begin{tikzpicture}[scale=0.8]
        \draw[thick] (0,0) node[below left] {$A$} -- (5,0) node[below right] {$B$} -- (2,3) node[above] {$C$} -- cycle;
        \draw[dashed, red] (2,3) -- (2,0) node[below, midway, right] {$h_c$};
        \node at (2.5, -0.5) {$c$};
    \end{tikzpicture}
    \caption{Троугао са основицом $c$ и висином $h_c$.}
\end{figure}
\end{frame}

\begin{frame}
\frametitle{Површина троугла 2/2}
Основна формула гласи:
$$P = \frac{a \cdot h_a}{2} = \frac{b \cdot h_b}{2} = \frac{c \cdot h_c}{2}$$
Где су $a, b, c$ странице троугла, а $h_a, h_b, h_c$ одговарајуће висине.
\end{frame}

\begin{frame}
\frametitle{Херонова формула (за радознале!)}
Постоји и формула која нам омогућава да израчунамо површину троугла ако знамо дужине све три његове странице. Ова формула се назива Херонова формула.
$$P = \sqrt{s(s-a)(s-b)(s-c)}$$
Где је $s$ полуобим троугла:
$$s = \frac{a+b+c}{2}$$
Ову формулу ћете детаљније учити у старијим разредима.
\end{frame}

% --- Секција: Површина четвороуглова ---
\section{Површина четвороуглова}

\begin{frame}
\frametitle{Површина правоугаоника и квадрата 1/2}
\textbf{Правоугаоник:}
\begin{itemize}
    \item Четвороугао са четири права угла.
    \item Наспрамне странице су једнаке.
\end{itemize}
Формула за површину правоугаоника:
$$P = a \cdot b$$
Где су $a$ и $b$ дужине суседних страница.
\end{frame}

\begin{frame}
\frametitle{Површина правоугаоника и квадрата 2/2}
\textbf{Квадрат:}
\begin{itemize}
    \item Правоугаоник са све четири једнаке странице.
\end{itemize}
Формула за површину квадрата:
$$P = a \cdot a = a^2$$
Где је $a$ дужина странице квадрата.
\end{frame}

\begin{frame}
\frametitle{Површина паралелограма и ромба 1/2}
\textbf{Паралелограм:}
\begin{itemize}
    \item Четвороугао чије су наспрамне странице паралелне и једнаке.
\end{itemize}
Формула за површину паралелограма:
$$P = a \cdot h_a = b \cdot h_b$$
Где је $a$ (или $b$) страница, а $h_a$ (или $h_b$) одговарајућа висина.
\begin{figure}
    \centering
    \begin{tikzpicture}[scale=0.7]
        \draw[thick] (0,0) node[below left] {$A$} -- (4,0) node[below right] {$B$} -- (5.5,2.5) node[above right] {$C$} -- (1.5,2.5) node[above left] {$D$} -- cycle;
        \draw[dashed, red] (1.5,2.5) -- (1.5,0) node[below, midway, left] {$h_a$};
        \node at (2, -0.4) {$a$};
    \end{tikzpicture}
\end{figure}
\end{frame}

\begin{frame}
\frametitle{Површина паралелограма и ромба 2/2}
\textbf{Ромб:}
\begin{itemize}
    \item Паралелограм са све четири једнаке странице.
\end{itemize}
Поред формуле за паралелограм ($P = a \cdot h_a$), за ромб важи и:
$$P = \frac{d_1 \cdot d_2}{2}$$
Где су $d_1$ и $d_2$ дијагонале ромба.
\end{frame}

\begin{frame}
\frametitle{Површина трапеза}
\textbf{Трапез:}
\begin{itemize}
    \item Четвороугао који има тачно један пар паралелних страница (основице).
\end{itemize}
Формула за површину трапеза:
$$P = \frac{a+b}{2} \cdot h \quad \text{или} \quad P = m \cdot h$$
Где су $a$ и $b$ паралелне основице, $h$ је висина трапеза, а $m = \frac{a+b}{2}$ је средња линија трапеза.
\begin{figure}
    \centering
    \begin{tikzpicture}[scale=0.7]
        \draw[thick] (0,0) node[below left] {$A$} -- (5,0) node[below right] {$B$} -- (3.5,2.5) node[above right] {$C$} -- (1.5,2.5) node[above left] {$D$} -- cycle;
        \draw[dashed, red] (1.5,2.5) -- (1.5,0) node[below, midway, left] {$h$};
        \node at (2.5, -0.4) {$a$};
        \node at (2.5, 2.8) {$b$};
    \end{tikzpicture}
\end{figure}
\end{frame}

% --- Секција: Бипартитни граф формула ---
\section{Везе између формула и променљивих}

\begin{frame}[fragile] % fragile је потребан због TikZ окружења
\frametitle{Формуле и њихове променљиве (Бипартитни граф)}
Погледајмо како су формуле за површине повезане са променљивима које се у њима појављују.

\begin{figure}
\centering
\begin{tikzpicture}[
    node distance=0.1cm and 2.5cm, % Размак између чворова
    every node/.style={font=\scriptsize} % Мањи фонт за све чворове
]
    % Чворови са формулама (лева страна)
    \node[formula] (F_tro) {$P = \frac{a \cdot h_a}{2}$};
    \node[formula, below=of F_tro] (F_pra) {$P = a \cdot b$};
    \node[formula, below=of F_pra] (F_kva) {$P = a^2$};
    \node[formula, below=of F_kva] (F_par) {$P = a \cdot h_a$};
    \node[formula, below=of F_par] (F_rom) {$P = \frac{d_1 \cdot d_2}{2}$};
    \node[formula, below=of F_rom] (F_tra) {$P = \frac{a+b}{2} \cdot h$};

    % Чворови са променљивима (десна страна)
    % Потребно је пажљиво позиционирати променљиве да граф буде прегледан
    \node[variable, right=4cm of F_tro] (var_a) {$a$};
    \node[variable, below=0.7cm of var_a] (var_ha) {$h_a$};
    \node[variable, below=0.7cm of var_ha] (var_b) {$b$};
    \node[variable, below=0.7cm of var_b] (var_d1) {$d_1$};
    \node[variable, below=0.7cm of var_d1] (var_d2) {$d_2$};
    \node[variable, below=0.7cm of var_d2] (var_h) {$h$};
    % Можда додати 's' за Херонову формулу ако се укључи у граф
    % \node[variable, below=of var_h] (var_s) {$s$};
    % \node[variable, below=of var_s] (var_c) {$c$}; % Ако је потребно за Херонову

    % Везе (гране)
    \draw[edge_style] (F_tro) -- (var_a);
    \draw[edge_style] (F_tro) -- (var_ha);

    \draw[edge_style] (F_pra) -- (var_a);
    \draw[edge_style] (F_pra) -- (var_b);

    \draw[edge_style] (F_kva) -- (var_a);

    \draw[edge_style] (F_par) -- (var_a);
    \draw[edge_style] (F_par) -- (var_ha);

    \draw[edge_style] (F_rom) -- (var_d1);
    \draw[edge_style] (F_rom) -- (var_d2);

    \draw[edge_style] (F_tra) -- (var_a);
    \draw[edge_style] (F_tra) -- (var_b);
    \draw[edge_style] (F_tra) -- (var_h);

    % Напомена: Можда ће бити потребно додатно подешавање позиција
    % променљивих чворова да би се избегло преклапање грана и да граф буде јаснији.
    % Ово је основни пример. За сложеније распоређивање, могу се користити
    % 'positioning' опције попут 'right=of var_a', 'below left=of var_b' итд.
    % или алгоритми за аутоматско распоређивање из `graphdrawing` библиотеке.
\end{tikzpicture}
\caption{Бипартитни граф који приказује везе између формула за површине и њихових променљивих. Лево су формуле, десно су променљиве.}
\end{figure}
\end{frame}

\begin{frame}
\frametitle{Пример везе: Површина троугла}
Размотримо формулу за површину троугла:
$$P = \frac{a \cdot h_a}{2}$$
\begin{itemize}
    \item Чвор са формулом "$P = \frac{a \cdot h_a}{2}$" је повезан са:
    \begin{itemize}
        \item Чвором променљиве "$a$" (страница троугла)
        \item Чвором променљиве "$h_a$" (висина која одговара страници $a$)
    \end{itemize}
\end{itemize}
Ово нам визуелно показује које величине треба да знамо да бисмо применили одређену формулу.
\end{frame}

\begin{frame}
\frametitle{Задаци за вежбу ✍️}
Сада ћемо решити неколико задатака користећи научене формуле.
\begin{enumerate}
    \item Израчунај површину правоугаоника чије су странице $a=5\text{cm}$ и $b=8\text{cm}$.
    \item Основица троугла је $12\text{cm}$, а одговарајућа висина је $7\text{cm}$. Колика је површина тог троугла?
    \item Колика је површина квадрата чија је страница $6\text{dm}$?
    \item Дијагонале ромба су $d_1=10\text{cm}$ и $d_2=14\text{cm}$. Израчунај његову површину.
    \item Основице трапеза су $a=15\text{m}$ и $b=9\text{m}$, а висина је $h=6\text{m}$. Нађи површину трапеза.
\end{enumerate}
\end{frame}

\begin{frame}
\frametitle{Закључак ✅}
Данас смо научили:
\begin{itemize}
    \item Формуле за израчунавање површина троуглова (основна).
    \item Формуле за израчунавање површина најважнијих четвороуглова:
    \begin{itemize}
        \item Правоугаоника и квадрата
        \item Паралелограма и ромба
        \item Трапеза
    \end{itemize}
    \item Како су формуле повезане са елементима фигура помоћу бипартитног графа.
\end{itemize}
Запамтите, разумевање ових формула је кључ за решавање многих геометријских проблема!
\end{frame}

\end{document}
